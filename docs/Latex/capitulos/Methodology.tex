\chapter{Metodología y Planificación}
\label{planificacion}

Este capítulo presenta la \textbf{metodología seguida para el desarrollo del proyecto}, así como la \textbf{planificación temporal y técnica}. En primer lugar, se expone el enfoque metodológico adoptado, justificando su idoneidad según el tipo de proyecto y sus objetivos. A continuación, se detallan las distintas etapas en las que se ha estructurado el desarrollo, desde la fase inicial de planificación hasta los resultados y conclusiones.
\\\\
También se hace referencia al concepto de \textbf{Mínimo Producto Viable}, fundamental para \textbf{establecer} una \textbf{primera versión funcional} del emulador que pueda ejecutarse y evaluarse con éxito. Por último, se describen las \textbf{herramientas empleadas} a lo largo del proyecto para el desarrollo de la aplicación.

\section{Metodología Aplicada}
La \textbf{metodología} adoptada para la planificación del proyecto es \textbf{ágil}, un enfoque que, si bien es comúnmente utilizado en proyectos colaborativos, resulta igualmente \textbf{eficaz en proyectos individuales}. La clave de esta metodología radica en la \textbf{gestión eficiente del tiempo}, permitiendo que cada tarea sea ejecutada dentro de un plazo bien definido, con el objetivo de maximizar los resultados y cumplir con los plazos establecidos.

\section{Etapas del Proyecto}
El proyecto está \textbf{dividido en distintas etapas} o iteraciones, cada una de las cuales ofrece la oportunidad de aprender y aplicar nuevos conocimientos relacionados con la emulación y el desarrollo. \textbf{Al final de cada iteración}, se lleva a cabo una \textbf{revisión} exhaustiva del progreso, lo que permite \textbf{identificar áreas de mejora o posibles errores}, minimizando así el tiempo perdido en futuras fases del desarrollo.
\\\\
Durante el proceso de desarrollo, las \textbf{tareas y objetivos} están en \textbf{constante evolución}, dado que se parte de una base de conocimientos limitada que se incrementa a medida que avanza el proyecto. Este enfoque implica que, en muchas ocasiones, lo que en un inicio parecía correctamente implementado debe ser revisado o incluso reestructurado, conforme se adquiere una comprensión más profunda de los desafíos técnicos involucrados.
\\\\
En la \textbf{siguiente tabla} se puede ver la \textbf{planificación estimada} con la que se pretende presentar y defender el proyecto a inicios de Junio de 2024 (C3). \textbf{No se tienen en cuenta posibles retrasos} por enfermedades, viajes u otros.

\begin{table}[H]
    \centering
    \begin{tabular}{|l|l|l|}
    \hline
    \textbf{Apartado} & \textbf{Tiempo estimado} & \textbf{Fecha límite} \\ \hline
    Agradecimientos, Objetivos y Justificación & 1 semana & 25 Agosto \\ \hline
    Planificación y Metodología & 2 semanas & 8 Septiembre \\ \hline
    Diseño & 4 semanas & 6 Octubre \\ \hline
    Aprendizaje & 1 semana & 13 Octubre \\ \hline
    Marco teórico & 3 semanas & 3 Noviembre \\ \hline
    Desarrollo & 5 semanas & 8 Diciembre \\ \hline
    Pruebas y validación & 2 semanas & 22 Diciembre \\ \hline
    Resultados & 1 semana & 29 Diciembre \\ \hline
    Conclusiones y trabajo futuro & 1 semana & 5 Enero \\ \hline
    Bibliografía y Referencias & 1 semana & 12 Enero \\ \hline
    \end{tabular}
    \caption{Planificación de contenidos y fechas límite}
\end{table}

Entrando en detalle en algunas etapas del proeycto:

\begin{itemize}
	\item \textbf{Diseño:} Antes de iniciar el desarrollo del emulador, es fundamental \textbf{tener claros los objetivos y el alcance del proyecto}. Se realizará un \textbf{análisis preliminar} de diversos emuladores existentes, lo cual permitirá obtener una visión más sólida y concreta de las \textbf{funcionalidades y desafíos} que se enfrentarán durante la implementación.

	\item \textbf{Aprendizaje:} En esta fase inicial, el enfoque será \textbf{comprender} en profundidad las \textbf{especificaciones técnicas} de la consola Game Boy, desde su CPU y PPU hasta la gestión de ciclos y sus interacciones con los componentes de hardware. Se realizarán \textbf{pruebas exploratorias}, cuyo propósito será construir una base sólida para el desarrollo posterior del emulador.
	
	\item \textbf{Desarrollo:} Esta es la \textbf{etapa central y más intensiva} del proyecto. Durante el desarrollo del emulador, se \textbf{implementarán las funcionalidades principales} como la emulación de la CPU, la gestión de gráficos, la sincronización de ciclos, y la integración con las interfaces gráficas en Android. Al mismo tiempo, se \textbf{generará la documentación técnica} detallada para acompañar el progreso y justificar las decisiones tomadas durante la implementación.
	
	\item \textbf{Revisión y Maquetación:} La fase final se centrará en la \textbf{revisión exhaustiva} tanto del emulador como de la documentación. Se \textbf{corregirán posibles errores} detectados, se \textbf{optimizará el rendimiento} del emulador, y se \textbf{pulirá la maquetación de la memoria}, asegurando que todo el trabajo cumpla con los estándares de calidad requeridos.
	
\end{itemize}
	
\section{Mínimo Producto Viable}

Lo normal en todo proyecto es que \textbf{ocurran imprevistos} que hagan al programador \textbf{perder más tiempo} en una tarea o incluso \textbf{paralizar por completo el proyecto}. Además, esto se junta con el hecho de que aquí no hay nadie que pueda ocupar nuestro puesto mientras ese problema se soluciona. Por esta razón, es importante tener en mente un \textbf{producto mínimo viable}, con el cual obtener un producto usable de en el tiempo disponible.

\section{Herramientas utilizadas}

El desarrollo de la aplicación se apoyará del uso de las siguientes tecnologías:

\begin{itemize}
    \item \textbf{Android Studio:} Se utilizará como editor de código principal, permitiendo la creación, prueba y depuración de la aplicación mediante emuladores de dispositivos Android.
    \item \textbf{Visual Studio Code:} Herramienta clave para la redacción y visualización de documentos en LaTeX, además de proporcionar un entorno gráfico para gestionar el control de versiones mediante Git.
    \item \textbf{Photoshop:} Herramienta principal de diseño utilizada para la creación de la interfaz gráfica de la aplicación.
    \item \textbf{Procreate:} Aplicación de diseño gráfico que se empleará para la creación de ilustraciones y recursos visuales específicos de la interfaz, complementando a Photoshop en tareas artísticas y detalladas.
\end{itemize}

\begin{figure}[H]
    \centering
    \includegraphics[width=0.7\textwidth]{include/images/herramientas.png}
    \caption{Herramientas utilizadas en el proyecto}
    \label{figure:tools}
\end{figure}
	
\cleardoublepage